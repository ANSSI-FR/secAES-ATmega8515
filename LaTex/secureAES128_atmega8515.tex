\documentclass{article}
\usepackage[utf8]{inputenc}
\usepackage[english,francais]{babel}
\usepackage{amsmath}
\usepackage{amssymb,amsfonts}
\usepackage{graphicx,xcolor}
\usepackage{subfigure}
\usepackage{float}
\usepackage{hyperref}
\usepackage{here}

\usepackage{fullpage}

%%%%%%%%%%%%%%%%%%%%%%%%%%%%%%%%%%%%%%%%%%%%%%%%%%%%%%%%%%%%%%%%%%%%%%%%%%%%%%%%%%%%%%%%%%%%%%%%%%%%%%%%%%%%%%%%%%%%%%%%%%%%%%%%%%%%%%%%%%%%%
% MACROS



%%%%%%%%%%%%%%%%%%%%%%%%%%%%%%%%%%%%%%%%%%%%%%%%%%%%%%%%%%%%%%%%%%%%%%%%%%%%%%%%%%%%%%%%%%%%%%%%%%%%%%%%%%%%%%%%%%%%%%%%%%%%%%%%%%%%%%%%%%%%%

\title{Secure AES128 on ATMEGA8515}

\author{Laboratory of Embedded Security, ANSSI}


%%%%%%%%%%%%%%%%%%%%%%%%%%%%%%%%%%%%%%%%%%%%%%%%%%%%%%%%%%%%%%%%%%%%%%%%%%%%%%%%%%%%%%%%%%%%%%%%%%%%%%%%%%%%%%%%%%%%%%%%%%%%%%%%%%%%%%%%%%%%%


\begin{document}

\pagestyle{plain}
\maketitle
\section{Introduction}

The members of the laboratoty of embedded security has developped two versions of AES128 for ATMEGA8515 device. The implementation codes are published for research and pedagogical puroposes only. The ATMEGA8515 component is not a secure component; in particular it works with an external clock and contains no hardware random generator. The information leakage is consequently particularly high and there is almost no jittering (traces' acquisition should therefore not suffer from toom much desynchronisation). To secure the implementation, it has been chosen to apply state of the art techniques: basic countermeasures for version 1 and improved countermeasures for version 2.


\section{Implementation Codes}

The AES128 implementations have been embedded in a minimal OS based on the open source OS SOSSE release under GPL v2.0 license (cf. \url{www.mbsks.franken.de/sosse/}). The AES128 API can therefore be executed through the iso7816 interface of the card. The list of available APDUs and their basic description are given in the table bellow (Table~\ref{table:apduCommands}). For information, a python script is also provided which launches an AES128 encryption et get the result (cf. \texttt{script-AES128-enc.py});

\begin{table}[H]
	\centering
	\begin{tabular}{l|l|llllll}
	Command              & Description                     & \multicolumn{6}{c}{Code}  \\
	\hline                                                  
	                      &                                 & CLA & INS & P1  & P2  & Lc/Le & Data              \\
	\hline
	\texttt{setKey}       & Load the $16$-bytes key   & 0x80& 0x10& 0x00& 0x00& 0x10  & 16 Bytes of the key    \\
	\texttt{getKey}       & Read the $16$-bytes key       & 0x80& 0x12& 0x00& 0x00& 0x10  &                      \\
	\texttt{setInput}     & Load the $16$-bytes message & 0x80& 0x20& 0x00& 0x00& 0x10  & 16 Bytes of the message \\
	\texttt{getInput}     & Read the $16$-bytes message    & 0x80& 0x22& 0x00& 0x00& 0x10  &                      \\
	\texttt{setMask}      & Load the $18$-bytes masking   & 0x80& 0x30& 0x00& 0x00& 0x12  & 18 Bytes of the mask  \\
	\texttt{getMask}      & Read the $18$-bytes masking       & 0x80& 0x32& 0x00& 0x00& 0x12  &                      \\
	\texttt{getOutput}    & Read the AES128 output   & 0x80& 0x42& 0x00& 0x00& 0x10  &                      \\
	\texttt{launchAES}    & Launch AES128 enc.   & 0x80& 0x52& 0x00& 0x00& 0x00  &                      \\
	\end{tabular}
	\caption{ADPU Commands}
	\label{table:apduCommands}
\end{table}

The project sources (OS + secure AES128) are provided in two different archives (one for each version, low security or high security): \texttt{Version1} and \texttt{Version2}. The AES128 implementations are essentially based on the papers \cite{AG01}, \cite{CJRR99}, \cite{FMPR10}, \cite{Wil01} and \cite{RPD09}. The source files contain many comments which should perfectly clarify the choices made to secure the algorithms and the implementation details. The (modified/simplified) sources of SOSSE still contain the original comments.


%%%%%%%%%%%%%%%%%%%%%%%%%%%%%%%%%%%%%%%%%%%%%%%%%%%%%%%%%%%%%%%%%%%%%%%%%%%%%%%%%%%%%%%%%%%%%%%%%%%%%%%%%%%%%%%%%%%%%%%%%%%%%%%%%%%%%%%%%%%%%
\bibliographystyle{abbrv}
\bibliography{biblio}




%%%%%%%%%%%%%%%%%%%%%%%%%%%%%%%%%%%%%%%%%%%%%%%%%%%%%%%%%%%%%%%%%%%%%%%%%%%%%%%%%%%%%%%%%%%%%%%%%%%%%%%%%%%%%%%%%%%%%%%%%%%%%%%%%%%%%%%%%%%%%

\end{document}
